%% LyX 2.1.4 created this file.  For more info, see http://www.lyx.org/.
%% Do not edit unless you really know what you are doing.
\documentclass[11pt,a4paper,english,thesis]{dcsbook}
\usepackage[utf8]{inputenc}
\setcounter{secnumdepth}{3}
\usepackage{color}
\usepackage{babel}
\usepackage{prettyref}
\usepackage{booktabs}
\usepackage{pmboxdraw}
\usepackage{graphicx}
\usepackage[unicode=true,pdfusetitle,
 bookmarks=true,bookmarksnumbered=true,bookmarksopen=true,bookmarksopenlevel=1,
 breaklinks=true,pdfborder={0 0 0},backref=false,colorlinks=true]
 {hyperref}
\hypersetup{
 urlcolor=linkcolor,linkcolor=linkcolor,citecolor=linkcolor}

\makeatletter

%%%%%%%%%%%%%%%%%%%%%%%%%%%%%% LyX specific LaTeX commands.
\pdfpageheight\paperheight
\pdfpagewidth\paperwidth

\providecommand{\LyX}{\texorpdfstring%
  {L\kern-.1667em\lower.25em\hbox{Y}\kern-.125emX\@}
  {LyX}}
\DeclareRobustCommand*{\lyxarrow}{%
\@ifstar
{\leavevmode\,$\triangleleft$\,\allowbreak}
{\leavevmode\,$\triangleright$\,\allowbreak}}
%% Because html converters don't know tabularnewline
\providecommand{\tabularnewline}{\\}

%%%%%%%%%%%%%%%%%%%%%%%%%%%%%% Textclass specific LaTeX commands.
\RequirePackage{dcslib}[2012/01/30]
 \newlength{\labelingxwidth}
 \newenvironment{labelingx}[2][]%
   {\ifthenelse{\equal{#2}{default}}%
      {\setlength{\labelingxwidth}{\leftmargin}}
      {\settowidth{\labelingxwidth}{#2}}
    \begin{labeling}[leftmargin=\labelingxwidth,#1]}
   {\end{labeling}}
 \newlist{lyxcodelist}{itemize}{1}
 \setlist[lyxcodelist]{listparindent=0pt,itemsep=0pt,parsep=0pt,partopsep=0pt}
 \providecommand{\lyxcodesetup}{}
 \newenvironment{lyxcode}
   {\par\begin{lyxcodelist}\normalfont\ttfamily\small\lyxcodesetup\item[]}
   {\end{lyxcodelist}}

%%%%%%%%%%%%%%%%%%%%%%%%%%%%%% User specified LaTeX commands.
%
%  $Id: thesis-template.lyx,v 1.7 2011/12/22 12:10:18 sobaniec Exp $
%

\@ifundefined{showcaptionsetup}{}{%
 \PassOptionsToPackage{caption=false}{subfig}}
\usepackage{subfig}
\makeatother

\usepackage{listings}
\renewcommand{\lstlistingname}{\inputencoding{latin9}Listing}

\begin{document}

\author{Wojciech Mioduszewski}


\title{Klasyfikacja danych opisanych za pomocą szeregów czasowych}


\date{Poznań, 2015}


\supervisor{dr inż. Jerzy Błaszczyński}

\maketitle


\frontmatter

\tableofcontents{}

\mainmatter


\chapter{Wstęp}

The Introduction may  be put \dcsemph{before} the \dcscode{\textbackslash{}mainmatter}
command which will disable numbering of this chapter while still adding
to the table of contents.


\paragraph{The goal and the scope of the thesis}


\chapter{Background}

The thesis can be structured using the following sectioning styles:


\section{Section}


\subsection{Subsection}


\subsubsection{Subsubsection}


\paragraph{Paragraph}


\subparagraph{Subparagraph}


\section{Inline formatting}

We suggest using \dcsemph{Insets}, like:
\begin{labelingx}{00.00.0000}
\item [{\dcsstrong{strong}}] for strong emphasizing some text.
\item [{\dcsemph{emph}}] for emphasizing some text.
\item [{\dcscode{Code}}] for formatting of names of modules, procedures,
class names, variables, etc.
\item [{\dcspath{path}}] for formatting of file names and directories,
like \dcspath{/usr/share/doc/packages/texlive-latex}. The names are
properly broken at the ends of lines. However, path names containing
special \LaTeX{} characters must be typeset using ERT and the \dcscode{\textbackslash{}dcspath}
command, e.g.\ \dcspath{sample_file}.
\item [{\dcskbd{kbd}}] for formatting of shortcuts, e.g.: \dcskbd{Ctrl-c}.
\item [{\dcscmd{cmd}}] for formatting system commands.
\item [{\dcsname{name}}] for formatting other special names.
\end{labelingx}

\section{Special characters}
\begin{enumerate}
\item Non-breaking space can be inserted using \dcskbd{Ctrl-space}. It
produces ``\dcscode{\textasciitilde{}}'' in \LaTeX{} code.
\item A normal, inter-word space can be inserted using \dcskbd{Ctrl-Alt-space}.
It produces ``\dcscode{\textbackslash{} }'' in \LaTeX{} code. This
type of space is useful for formatting spacing after dots, e.g.\ here.
By default \LaTeX{} produces here a longer space used for separating
whole sentences.
\item A thin space can be produced by \dcskbd{Ctrl-Shift-space}, e.g.\,here.
It produces ``\dcscode{\textbackslash{},}'' in \LaTeX{} code.
\item Sentence-ending space can be inserted using \dcskbd{Ctrl-.}, which
produces ``\dcscode{\textbackslash{}@.}'' in \LaTeX{} code. This
type of space is useful in sentences ending with a capital letter.
In such cases \LaTeX{} recognizes the last word as a acronym and places
a regular inter-word space instead of inter-sentence space. Consider
the following example:

\begin{quote}
\emph{This can be achieved by using HTTP\@. This protocol...}
\end{quote}
\item Hyphenation indicator can be inserted using \dcskbd{Ctrl-~--}, which
is used for marking possible places of hyphenation, e.g.\ de\-mo\-cra\-cy.
\end{enumerate}

\section{Figures}

The figures should be put in floats, like Fig.~\ref{fig:ex-fig1}.
You can also reference figures using \dcsname{prettyref} package
like this: \prettyref{fig:ex-fig1}.

\begin{figure}[tbph]
\begin{centering}
\includegraphics[width=0.4\textwidth]{logopp/logopp-jasne-czarne}
\par\end{centering}

\caption{Example figure \label{fig:ex-fig1}}
\end{figure}


It is possible to combine several pictures inside one float. Just
insert a float inside a float. See Fig.~\ref{fig:ex-fig2} for example.
Please note the horizontal spacing between subfigures.

\begin{figure}[tbph]
\centering{}\hfill{}\subfloat[The first subfigure]{\begin{centering}
\includegraphics[width=0.3\textwidth]{logopp/logopp-jasne-czarne}
\par\end{centering}

}\hfill{}\subfloat[The second subfigure]{\begin{centering}
\includegraphics[width=0.3\textwidth]{logopp/logopp-jasne}
\par\end{centering}

}\hfill{}\mbox{}\caption{Example figure \label{fig:ex-fig2}}
\end{figure}



\section{Tables}

Tables should have captions above like Table~\ref{tab:ex-tab}. Use
small sans-serif fonts inside tables.

\begin{table}[h]
\caption{Example table \label{tab:ex-tab}}


\textsf{\small{}}%
\begin{tabular}{lcc}
\toprule 
\textsf{\textbf{\small{}Column 1}} & \textsf{\textbf{\small{}Column 2}} & \textsf{\textbf{\small{}Column 3}}\tabularnewline
\midrule 
\textsf{\small{}One} & \textsf{\small{}1} & \textsf{\small{}4}\tabularnewline
\midrule 
\textsf{\small{}Two} & \textsf{\small{}2} & \textsf{\small{}5}\tabularnewline
\midrule 
\textsf{\small{}Three} & \textsf{\small{}3} & \textsf{\small{}6}\tabularnewline
\bottomrule
\end{tabular}{\small \par}
\end{table}


\newpage{}


\section{Source code examples}

There are a few different methods of including sample codes:
\begin{enumerate}
\item Using standard \dcscode{\LyX{}-Code} style:

\begin{lyxcode}
\#include~<stdio.h>

~

int~main()~\{

~~printf(\textquotedbl{}Hello~world!\textbackslash{}n\textquotedbl{});

~~return~0;

\}
\end{lyxcode}

Note 1: Empty lines must contain at least one single space to remain
visible.


Note 2: There is no way to activate automatic syntax highlighting
inside \dcscode{\LyX{}-Code}. However, you can use normal inline
formatting inside.


Note 3: \dcscode{Lyx-Code} can contain special characters, so it
can be used to produce some ASCII art, e.g.:
\begin{lyxcode}
┌──────────┐~~~~╔══════════╗

│~~Test~1~~├────╢~~Test~2~~║

└──────────┘~~~~╚══════════╝
\end{lyxcode}
\item By inserting \dcsemph{Program Listing}:


\inputencoding{latin9}\begin{lstlisting}[language=C,tabsize=2,xleftmargin=2em]
#include <stdio.h>

int main() {
  printf("Hello world!\n");
  return 0;
}
\end{lstlisting}
\inputencoding{utf8}


Note: By default the \dcscode{lstlisting} environment does not add
any left margin. You can change it by adding \dcscode{xleftmargin}
in the \dcsemph{Settings} \lyxarrow{} \dcsemph{Advanced} dialog
box, e.g.:


\inputencoding{latin9}\begin{lstlisting}[tabsize=2,xleftmargin=6em]
procedure sayHello()
\end{lstlisting}
\inputencoding{utf8}

\item By inserting \LaTeX{} Code (ERT block) and using \dcscode{codeblock}
environment:


\begin{codeblock}[language=c]
#include <stdio.h>
 
int main() {
  printf("Hello world!\n");
  return 0;
}
\end{codeblock}

\item The \dcsname{listings} package can produce floats by itself. See
Listing.~\ref{lst:hello} for example.


\inputencoding{latin9}\begin{lstlisting}[caption={The Hello World program in C},float,label={lst:hello},language=C,style=linesbg,tabsize=2]
#include <stdio.h>

int main() {
  printf("Hello world!\n");
  return 0;
}
\end{lstlisting}
\inputencoding{utf8}

\item And finally, You can include code from external file:


\codeincp[language={[latex]tex},lastline=9]{thesis-template-latex.tex}

\end{enumerate}

\section{Math}

Can be put inline like this: $S=\sum_{i=1}^{i=K}x_{i}^{2}$ or in
dedicated lines:
\[
S=\sum_{i=1}^{i=K}x_{i}^{2}
\]
The equations can be also numbered like equation~\ref{eq:est}. 
\begin{equation}
s=\sqrt{\frac{1}{n-1}\sum_{i=1}^{n}\left(x_{i}-\overline{x}\right)^{2}}\label{eq:est}
\end{equation}



\section{Algorithms}

Use \dcsname{dcsalg} package or directly \dcsname{algorithmicx}
package.


\section{Bibliography}

The bibliography can be included in the thesis like in this case.
You can then cite the publications like this~\cite{sop}. The other
(more professional) solution is to use Bib\TeX{}\@. See \dcsemph{\LyX{} User's Guide}
for details.


\chapter{Concept and Design of the System}


\chapter{Implementation}


\chapter{Performance Evaluation}


\chapter{Conclusions}

\appendix

\chapter{Users Guide}

\backmatter
\begin{thebibliography}{Odnośniki}
\bibitem{sop}A.~Tanenbaum. \emph{Operating Systems Design and Implementation}.
Prentice Hall, 2006.\end{thebibliography}

\end{document}
